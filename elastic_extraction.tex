%%%%%%%%%%%%%%%%%%%%%%%%%%%%%%%%%%%%%%
% Contribution to IKP annual report 2021
%%%%%%%%%%%%%%%%%%%%%%%%%%%%%%%%%%%%%%
\documentclass[fleqn,twocolumn,a4paper]{ikpar}
\usepackage{hyperref}
\usepackage{paralist}
\usepackage{times,mathptm}
\usepackage{graphicx}
\usepackage{caption}
\usepackage{subfig}
\usepackage{gensymb}
\pagestyle{empty}

% standadized SI units
\usepackage[per-mode=symbol, range-units=single, binary-units=true]{siunitx}
\DeclareSIUnit\clight{\text{\ensuremath{c}}} % redefine light speed symbol
\DeclareSIUnit\momentum{\GeV\per\clight} % momentum in GeV/c
\DeclareSIUnit\tmom{(\momentum)^2} % 4-momentum squared
\DeclareSIUnit\atom{\text{atoms}} 
\DeclareSIUnit\event{\text{events}} 

% \addtolength{\topmargin}{-0.2cm}
% \addtolength{\textheight}{0.2cm}
\begin{document}
\parindent=0pt
\frenchspacing

\title{{\bf
    Extraction of elastic scattering events in KOALA
}}
\author{Yong Zhou
}

\maketitle

%%%% Overall energy spectrums, background problem
Precise extraction of the elastic scattering event counts are critical for the
determination of the differential cross section in KOALA.
The reconstructed energy spectrums of the recoil detector show clear pattern of
elastic scattering energy variation along the beam axis, as shown in Fig. \ref{fig:energy_vs_strips}.
The elastic peaks are well seperated from the low-energy background at large
recoil angles, while they are hard to be identified at small recoil angles.
Thus, different techniques are needed to extract the elastic events from the
background for strips at different recoil angles.
Three methods are studied and presented in this report and the consitence of the results
from these methods are discussed.
\begin{figure}[!htb]
	\includegraphics[width=0.48\textwidth]{./energy_vs_strips.png}
  \caption{Energy spectrums of all strips of the recoil detector obtained at \SI{2.2}{\momentum}.}
  \label{fig:energy_vs_strips}
\end{figure}

\par
\medskip

%%%%% Method 1: Combined fit 
For all strips of Ge1/Ge2 and most strips of Si2, a combined fit with
a three-components background model and an elastic peak model is used to extract
the elastic events.
The elastic peak is described by the so-called crystal-ball function\cite{r1}, which
is composed of a Gaussian core and two power-law tails on both sides of the core.
The three components in the background model are: 1) a
fast-decreasing exponetial component which has very high yield at low energy; 2)
a slow-decreasing exponetial componet which extends to very high energy; 3) a
minimum-ionizing-particle (MIP) component which is described by the Landau
distribution.
A typical result using this fit model is shown in Fig. \ref{fig:combined_fit}.
The MIP events are mainly generated by the pions from the inelastic interaction,
which have the most probable energy deposit of \SI{0.37}{\MeV} in Si1/Si2,
\SI{2.2}{\MeV} in Ge1 and \SI{7.0}{\MeV} in Ge2.
The accuracy of this method deteriates when the elastic peak is close to the MIP's
peak, and both peaks are embeded among the high low-energy background.
In this case, it's hard to estimate the fitting parameters of each component and large
error exists in determining the fraction of each background component.
\begin{figure}[!htb]
	\includegraphics[width=0.45\textwidth]{./combined_fit.png}
  \caption{An example of the combined fit of the total energy spectrum for strips at large recoil angles.}
  \label{fig:combined_fit}
\end{figure}

\par
\medskip

%%%%% Method 2: TOF-E selection
For recoil strips at small recoil angles, the low-energy background can be effectively supressed by selecting the elastic
events using the TOF-E relation of the recoil proton.
The energy spectrums from three strips at different recoil angles are shown in
Fig. \ref{fig:coulomb_cb2_fit},  before and after the TOF-E selection.
The only remaining background after this selection comes from the elastic scattering of the residual gas in the beam pipe.
The shape of this background is described well by the Coulomb elastic scattering cross
section, due to the uniform distribution of the residual gas and the rapid decreasing of
the elastic scattering cross section beyond Coulomb region.
Thus, the extraction of the elastic events from the target body is carried out
using the combined fit of a Coulomb elastic scattering formula and the crystal-ball
function.
The results of this fit for the three example strips are also shown in Fig. \ref{fig:coulomb_cb2_fit}.
However, this method is constrained by the limited acceptance of the forward detector:
1) below the lower limit, the full target body can't be covered as shown in Fig.
\ref{fig:coulomb_cb2_fit} (a);
2) beyond the upper limit, the full beam profile can't be covered as shown in
Fig. \ref{fig:coulomb_cb2_fit} (c).
Only a small number of strips on Si1 and Si2 can use this method to exract all
elastic scattering events from the beam-target interaction.
\begin{figure}[!htb]
	\includegraphics[width=0.45\textwidth]{./coulomb_cb2_fit.png}
  \caption{The energy spectrums of TOF-E selected events (black dots) from strips at three
    recoil angles. The black lines are from all events and the grey area are the
  supressed background events. }
  \label{fig:coulomb_cb2_fit}
\end{figure}

\par
\medskip

%%%%% Method 3: Background substraction
For strips on Si1 and Si2 which are neither fully covered by the forward detector nor
the elastic peak of which is well seperated from the MIP's peak and lies
far away from the low-energy background, the third method is used to
extract the elastic events.
First, a background model is extracted from the strips which are fully covered by the forward detector.
The elastic peaks in these strips are already removed using the second method, thus generating 
pure inelastic background spectrums which are used as the background model
for analyzing other strips in the same sensor.
This background model is combined with the crystal-ball function to fit the full
energy spectrum and extract the elastic events.
An example of the fit result for the same channel in Fig. \ref{fig:coulomb_cb2_fit}
(c) is shown in Fig. \ref{fig:bkg_vs_tofe}, in which the fitted background component is substracted. 
Comparing to the TOF-E selected elastic peak, the full elastic peak can now be extracted.
This method can be applied on all strips on Si1/Si2.
However, the systematic error becomes larger for strips located far away from
the strips which are used for the extraction of the background model.
\begin{figure}[!htb]
	\includegraphics[width=0.45\textwidth]{./bkg_vs_tofe.png}
  \caption{Comparison of the elastic peak extracted using method 3 and the TOF-E
    selected elastic peak using method 2.}
  \label{fig:bkg_vs_tofe}
\end{figure}

\par
\medskip

%%%%% Discussion: Consistence between 3 methods: 3 VS 1, 3 VS 2
The consitence of different extraction methods are checked by comparing the event
number under the extracted elastic peak of the same recoil strip.
Two groups of strips from Si1 and Si2 respectively are selected for the
comparison between method 2 and 3, and one group of strips from Si2 are selected
for the comparison between method 1 and 3.
The results are summarized in Fig. \ref{fig:extraction_consistence}.
The difference of the extracted event number are all smaller than \SI{0.6}{\percent},
showing excellent consitency of the three methods.
The large discrepancy at low and high recoil energies between method 2 and
method 3 on Fig. \ref{fig:extraction_consistence} (a) is due to the limited
acceptance of the forward detector, which is whithin expectation.
The fully-covered strips on Si1 and Si2 can also be identified based on this comparison.
\begin{figure}[!htb]
	\includegraphics[width=0.45\textwidth]{./comparison_methods.png}
  \caption{Consistence between different extraction
    methods: a) the ratio of elastic scattering event counts between method 2 and
    method 3, the shading regions showing the fully-covered strips on Si1
    (black) and Si2 (red) respectively; b) the ratio of elastic scattering event counts between method 3 and
    method 2.}
  \label{fig:extraction_consistence}
\end{figure}

\par
\medskip

%%%%% Conclusion
Three methods of extracting the elastic scattering event peak from the total
energy spectrum are studied and compared.
The cystall-ball function is adopted as the reponse function of single recoil strip to the
elastic scattering events from the target body.
The three methods differs by processing the background components with different techniques.
Although these methods are consitent with each other, it's found that they are best suited for
processing strips located at different recoil angles to achieve the best accuracy
and smallest error.
No single method is suitable for processing all recoil strips.
The extracted elastic event counts from these methods can be used as
the essential input data set for the determination of the differential cross-section in the following analysis.

\par
\medskip

%----------------------------------------------------------------
\begin{thebibliography}{99}
\bibitem{r1} J. E. Gaiser, Appendix-F Charmonium Spectroscopy from Radiative Decays of the J/Psi and Psi-Prime, Ph.D. Thesis
\end{thebibliography}
\end{document}

