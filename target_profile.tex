%%%%%%%%%%%%%%%%%%%%%%%%%%%%%%%%%%%%%%
% Contribution to IKP annual report 2021
%%%%%%%%%%%%%%%%%%%%%%%%%%%%%%%%%%%%%%
\documentclass[fleqn,twocolumn,a4paper]{ikpar}
\usepackage{hyperref}
\usepackage{paralist}
\usepackage{times,mathptm}
\usepackage{graphicx}
\usepackage{float}
\usepackage{caption}
\usepackage{subfig}
\usepackage{gensymb}
\pagestyle{empty}

% standadized SI units
\usepackage[per-mode=symbol, range-units=single, binary-units=true]{siunitx}
\DeclareSIUnit\clight{\text{\ensuremath{c}}} % redefine light speed symbol
\DeclareSIUnit\momentum{\GeV\per\clight} % momentum in GeV/c
\DeclareSIUnit\tmom{(\momentum)^2} % 4-momentum squared
\DeclareSIUnit\atom{\text{atoms}} 
\DeclareSIUnit\event{\text{events}} 

% \addtolength{\topmargin}{-0.2cm}
% \addtolength{\textheight}{0.2cm}
\begin{document}
\parindent=0pt
\frenchspacing

\title{{\bf
    Determination of the cluster target density profile in KOALA
}}
\author{Yong Zhou
}

\maketitle

%%%% Background, problem description, topic and results of this report
KOALA aims to measure the differential cross section of (anti)proton-proton elastic
scattering over a wide $|t|$ range from \SIrange{0.0008}{0.1}{\tmom}.
In the initial setup, in which only the recoil detector is commissioned, the high yield of low-energy
background detererioates the elastic peak identification at small recoil angles and limits the lower measurement range of KOALA.
The forward detector, which is made from fast scintillators, has been built and
commissioned in 2019 to help supressing the low energy background based on TOF-E
relation of the recoil protons.
The design of the forward detector has been veried in the following test runs and energy spectrums from pure elastic scattering events can now be obtained after the TOF-E selection.
Due to the lower limit of the acceptance of the forward detector, the spectrums
of strips at very small recoil angles start to lose some events from the
target body beyond a threshold angle.
However, it is found that this phenomenon appears at a larger recoil angle than the design goal.
This is caused by the unexpected large thickness of the cluster target along the beam axis.
Thus, it's important to get more accurate information about the target profile.
In this report, a new method of determining the density profile of the cluster
target in KOALA is described, and the result from the 2019 beam test is presented.
The implication of measured target profile on the lower limit of KOALA measurement range is also discussed.

\par
\medskip

%%%%% Description of the method
The target thickness can be ignored at large recoil angles, since the strips are
far away from the interaction point.
This assumption does not hold for strips located close to the interaction point.
Three scenerios are depicted in Fig. \ref{fig:target_density_determination} to
show the variation of the measured energy spectrums with respect to smaller
recoil angles.
The acceptance of the forward detector covers the fully target body in the
beginning (Fig. \ref{fig:target_density_determination} (1)), it gradually losts part of the target body
(Fig. \ref{fig:target_density_determination} (2)), then the peak, and finally only covers
the tail of the target body (Fig. \ref{fig:target_density_determination} (3)).
In the end, it losts the target body completely and only measures the residual
gas contribution.
\begin{figure*}[h!]
  \centering
	\includegraphics[width=0.8\textwidth]{./target_density_determination.png}
  \caption{Three scenerios showing the variation of recorded elastic energy
    spectrums measured by three strips at different recoil angles: (1) the full
    target body is within the acceptance of the
    forward detector; (2) part of the target body is not within the
    acceptance, but target center is still covered; (3) the target center is out of the
    acceptance, only the tail of the target profile is recorded. The first row
    shows the target profile shape and the second row is the measured energy
    spectrum at each recoil angle. The blue bar indicates the region covered by
    the forward detector and the red dot indicates the reference energy bin and
    its corresponding position in the target profile.}
  \label{fig:target_density_determination}
\end{figure*}

The number of TOF-E selected elastic scattering events $N_{elastic}$ recorded on a single recoil strip has the
following relation with other experiment parameters,
\begin{equation}
  N_{elastic} = \epsilon_{DAQ}\cdot l_{beam}\cdot\rho_{target}\cdot\sigma_{elastic}\cdot\epsilon_{acceptance}
\end{equation}
where $\epsilon_{DAQ}$ is the DAQ efficiency, $l_{beam}$ is the beam intensity, $\rho_{target}$ is the target density,
$\sigma_{elastic}$ is the cross section of elastic scattering and
$\epsilon_{acceptance}$ is the acceptance of the forward detector.
$l_{beam}$ and $\epsilon_{DAQ}$ are constant in the same run.
To determine $\rho_{target}$, the last two items $\sigma_{elastic}$ and
$\epsilon_{acceptance}$ need to be fixed.
$\sigma_{elasitc}$ is determined by the energy of the recoil proton, thus can be
fixed by choosing the same energy value as reference.
$\epsilon_{acceptance}$ varies with the recoil angle as well as the beam size.
Simulation shows that strips at the lower range of the forward coverage are
guaranteed to be fully covered as long as the width of the beam profile is
smaller than \SI{10}{mm}, see Fig. \ref{fig:fwd_acceptance}.
Since the strip corresponds to a specific recoil angle which in turn
can be converted to an energy value, this indicates that the elastic events with recoil energy close to the lower
limit of the forward detector's acceptance range are guaranteed to be fully covered and has $\epsilon_{acceptance} = 1$.
Thus, the event number in the energy bins close to the edge of the energy
spectrum changes only with $\rho_{target}$ and can be used directly as the relative measurement of the target density.
Since the reference energy can be converted back to a fixed position in front of the
strip center, the position distribution of the target profile along the beam axis is
determined by recording the event counts of the same energy bin in all strips.
This process is also shown in Fig.\ref{fig:target_density_determination}, where \SI{120}{keV} with bin width of \SI{10}{keV} is selected as the reference.
\begin{figure}[htb!]
  \centering
	\includegraphics[width=0.5\textwidth]{./fwd_acceptance.png}
  \caption{Acceptance of the forward detector from the simulation using a square
    beam with \SI{10}{\mm} width. The blue blocks are the hit map on the forward
    detector plane for events hitting the recoil sensors. The red square indicates
    the size of the forward scintillators.}
  \label{fig:fwd_acceptance}
\end{figure}


% The event count in the same energy bin close to the lower edge of the forward
% fully-covered region is selected as the reference for the density determination.
Fig. \ref{fig:spectrums} shows the elastic energy spectrums recorded by different strips of Si1 at beam momentum \SI{2.2}{\momentum}.
The observed lower edge of the forward acceptance is about \SI{100}{\keV} while the
trigger threshold is about \SI{50}{\keV}.
The lower edge is in accordance with the calculated range of the
fully-covered recoil energy by the forward detector at \SI{2.2}{\momentum},
which is \SIrange{0.096}{1.56}{\MeV}.
Due to the full coverage, the event counts in the energy bin close to the lower edge (for example
\SI{120}{\keV}) acts as the meter of the target density at the position
corresponding to this energy bin.
Since strips are located along the beam axis, both before and after the target
chamber center, the aggregation of the measurements on all the strips provide a full scan of the target denstiy profile along the beam axis.
The strips are organized into two groups in Fig. \ref{fig:spectrums}: 1) the
target peak density is not covered by the strips
before Si1\_18 and they measures the target profile before the peak density; 2) the target
peak gradually shows up on strips after Si1\_18 and they measures the profile
after the peak density.

\par
\medskip

%%%%% Results of the method
Two independent methods can be used to determine the position distribution of the density profile.
The first method is based on the strip position of the ideal geometry model.
The reference energy bin is convered to a position before the strip center (for
\SI{120}{\keV}, it converts to \SI{10}{\mm} at \SI{2.2}{\momentum}), which in turn
gives the absolute position of this profile slice in the reference frame of the
ideal geometry model.
The second method is based on the energy difference of the reference energy bin
and the elastic peak.
The elastic energy peak corresponds to the target density peak.
The energy difference can be converted to the relative position of this profile
slice against the target profile peak.
This method is limited by the coverage of the elastic and only gives the density distribution after the target peak.
The comparison between these two results is used to align the the recoil
detector to the target center.
\begin{figure}[htbp]
  \centering
	\includegraphics[width=0.35\textwidth]{./target_density_result.png}
  \caption{Acceptance of the forward detector. a \SI{10}{\mm} square beam size.}
  \label{fig:target_density_result}
\end{figure}

\par
\medskip

%%%%% Discussion of the results
Fig. \ref{fig:result} shows the target density profile measured at
\SI{2.2}{\momentum}.
The target density is almost symmetric against the peak value.
The main body of the target is fitted with a double-gaussian, in which the
narrow component has $\sigma$ of \SI{2}{mm} and the wide one has $\sigma$ of \SI{5.6}{mm}.
The overall FWHM of the target distribution is about \SI{6.9}{mm}.
A flat residual gas distribution over a wide region is also observed.
Although the dendity of the residual gas is about 200x lower than the peak
value, they introduce an unnegligible elastic background at small recoil angle.

The measured target density distribution is wider than required design value and
explains the deteriation observed at small recoil angles.
With the newly measured target density profile, it's now possible to unfold the the
measured spectrum at very small recoil angles.
This opens the possibility of extending the range of measurable $|t|$ to a lower limit.

\par
\medskip

%----------------------------------------------------------------
\begin{thebibliography}{99}
\bibitem{r2} Y. Zhou, H. Xu, IKP Annual Report 2018
\bibitem{r4} \url{https://root.cern.ch}
\end{thebibliography}
\end{document}

